\subsubsection{Output Configuration Provider}
\label{sec:outputConfigurationProvider}

In JSF applications all the web related content is normally placed under
\texttt{./WebContent} instead of \texttt{./src-gen}, which is mostly used as
output for generated java artifacts. We want to adapt to the web applications
structure and seperate generated java classes and JSF artifacts from each other. For that purposes add
a class called \texttt{JsfOutputConfigurationProvider.java} derived from
\texttt{org.eclipse.xtext.generator.OutputConfigurationProvider} 
\footnote{\url{http://xtextcasts.org/episodes/15-output-configurations}} within package
\texttt{org.eclipse.xtext.example.ql.generator}.The new provider adds an
additional \newline \texttt{OutputConfiguration} for the ouput directory
\texttt{./WebContent} as shown in the listing below.

\begin{lstlisting}[language=Java]
package org.eclipse.xtext.example.ql.generator;

import java.util.Set;

import org.eclipse.xtext.generator.OutputConfiguration;
import org.eclipse.xtext.generator.OutputConfigurationProvider;

public class JSFOutputConfigurationProvider extends OutputConfigurationProvider {

  public final String WEB_CONTENT = "WebContent";

  /**
   * @return a set of {@link OutputConfiguration} available for the generator
   */
  public Set<OutputConfiguration> getOutputConfigurations() {
    Set<OutputConfiguration> outputConfigurations = super
        .getOutputConfigurations();

    OutputConfiguration webContent = new OutputConfiguration(WEB_CONTENT);
    webContent
        .setDescription("Read-only Output Folder for web generated application artifacts");
    webContent.setOutputDirectory("./WebContent");
    webContent.setOverrideExistingResources(true);
    webContent.setCreateOutputDirectory(true);
    webContent.setCleanUpDerivedResources(true);
    webContent.setSetDerivedProperty(true);
    outputConfigurations.add(webContent);

    return outputConfigurations;
  }
}
\end{lstlisting}

The interface \texttt{OutputConfiguration} provides several options to configure
the behavior of the so called \texttt{Outlet}. We use the defaults as in
\texttt{OutputConfigurationProvider} except its name, description and outputDirectoy.

Our \texttt{JsfOutputConfigurationProvider} can be bound in the
\texttt{QlDslRuntimeModule} by overriding/extending the method
\texttt{configure}.

\begin{lstlisting}[language=Java]
@Override public void configure(Binder binder) {
    super.configure(binder);
    binder.bind(IOutputConfigurationProvider.class)
        .to(JSFOutputConfigurationProvider.class).in(Singleton.class);
  }
 \end{lstlisting}
 
After this step we can refer to the additional \texttt{OutputConfiguration} in
generators by use of the constant \texttt{WEB\_CONTENT} defined in class
\texttt{JsfOutputConfigurationProvider}.