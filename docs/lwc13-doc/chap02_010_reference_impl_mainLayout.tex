\subsubsection{Main Layout}
\label{subsec:referenceMainLayout} 

The logical entry to the web application is the \texttt{welcome-file}
(\texttt{WebContent/index.xhtml}) declared in
\texttt{WebContent/WEB-INF/web.xml}. The \texttt{index.xhtml} will later be
helpful to integrate the generated artifacts with the web applications layout by
use of JSF xhtml templating.
\footnote{\url{http://docs.oracle.com/javaee/6/javaserverfaces/2.1/docs/vdldocs/facelets/}}

\begin{lstlisting}[language=HTML]
<?xml version='1.0' encoding='UTF-8' ?>
<!DOCTYPE html PUBLIC "-//W3C//DTD XHTML 1.0 Transitional//EN" 
    "http://www.w3.org/TR/xhtml1/DTD/xhtml1-transitional.dtd">
<html xmlns="http://www.w3.org/1999/xhtml"
  xmlns:h="http://java.sun.com/jsf/html"
  xmlns:ui="http://java.sun.com/jsf/facelets">
<body>

  <ui:composition
    template="/resources/default/templates/defaultLayout.xhtml">
    <ui:define name="content">
      Hello World!
    </ui:define>
  </ui:composition>

</body>

</html>
\end{lstlisting}
 
Subpages of the application should use the \texttt{index.xhtml} placed in
\texttt{WebContent/} itself as their template and overwrite the content section
with custom output by the same pattern.

\begin{lstlisting}[language=HTML] 
	<ui:composition template="/index.xhtml">
  		<ui:define name="content">
  		...
  		</ui:define>
 	</ui:composition>
\end{lstlisting}

Everything between the opening and closing \texttt{facelet:define} tag within
the subpage will be passed into a \texttt{facelet:insert} section defined in the
template or one of its parent templates when the html output is rendered by the
JSF framework.

\paragraph{Default template}
$\;$ \\The reference application is shipped with a default template placed in
\newline \texttt{WebContent/resources/default/templates/defaultLayout.xhtml}. Its a very
simple one which privides mainly 3 sections (header, content, footer) which can
be overloaded by the layouts children.

To change the layout it is just necesarry to change the main layout reference in
the attribute \texttt{template} of the \texttt{facelets:composite} component.
The current web application urgently needs a defined facelets:insert section
with name 'content' for proper composition which is declared in
\texttt{/resources/default/templates/defaultLayout.xhtml}.

 
The \texttt{index.xhtml} creates the main composition of the applications layout
and pages. 
\begin{lstlisting}[language=HTML]
<!DOCTYPE html PUBLIC "-//W3C//DTD XHTML 1.0 Transitional//EN" 
          "http://www.w3.org/TR/xhtml1/DTD/xhtml1-transitional.dtd">
<html xmlns="http://www.w3.org/1999/xhtml"
	xmlns:h="http://java.sun.com/jsf/html"
	xmlns:ui="http://java.sun.com/jsf/facelets">
<h:head>
	<title><ui:insert name="title">LWC 2013 Xtext</ui:insert></title>
</h:head>
<body>
	<div id="header">
		<ui:insert name="header">
			<ui:include src="/resources/default/templates/header.xhtml" />
		</ui:insert>
	</div>
	<div id="content">
		<ui:insert name="content">
    	Content area. Compose by use of tag facelet:define & name="content".
  </ui:insert>
	</div>
	<div id="footer">
		<ui:insert name="footer">
			<ui:include src="/resources/default/templates/footer.xhtml" />
		</ui:insert>
	</div>
</body>
</html>
\end{lstlisting}

